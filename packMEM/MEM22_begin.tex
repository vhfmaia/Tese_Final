%%RM2022
%% requires \usepackage{xstring}

\newcommand{\tipologiaDPE}[1]{%
\IfStrEqCase{#1}{
    {D}{Dissertação apresentada}
    %{P}{Relatório de Projeto apresentado}
    {P}{Trabalho de Projeto apresentado}
    {E}{Relatório de Estágio apresentado}}
    [tipologia errada]
}
%%RM2022


% -------------------------------------------------------------
% Definição da capa (com e sem figuras)
% -------------------------------------------------------------
\iffalse            % \iffalse: capa sem imagem, \iftrue: capa com imagem
   \TitlePage                             % Capa da tese com uma imagem
%      \GRID                              % Desenha grelha rectangular de 3mm
      \HEADER{\BAR\FIG{\includegraphics{./Imagens/nice.pdf}}}    % Imagem para a capa
             {\thesisyear}
      \TITLE{\authorname}
            {\thesistitle \vspace*{5mm} \thesistitleen}
   \EndTitlePage
\else
   \TitlePage    % Capa da tese sem imagens, com citação
%      \GRID                              % Desenha grelha rectangular de 3mm
      \HEADER{\BAR\FIG{\begin{minipage}{100mm}   % Largura da caixa da citação (<120mm)
             {}    % Citação 1. Colocar {} para retirar citação
         \begin{flushright} 
             {}  % Citação 2. Colocar {} para retirar citação
         \end{flushright}
      \end{minipage}}}
          {\thesisyear}
    \IfStrEqCase{\lang}{
    {E}{
     \TITLE{\authorname}
      \thesistitleen 
      \vspace*{5mm}
     \TEXT{} {\textsf{\Large{\thesistitle}}}
    }
    {P}{
   \TITLE{\authorname}
      \thesistitle
      \vspace*{5mm}
      \TEXT{} {\textsf{\Large{\thesistitleen}}}
    }
    }
 
\EndTitlePage
\fi
\titlepage\ \endtitlepage    % Página de verso em branco


% -------------------------------------------------------------
% Definição da capa interior
% -------------------------------------------------------------

\TitlePage
%   \PGRID                                    % Desenha grelha rectangular de 3mm
   \HEADER{}{\thesisyear}
   \IfStrEqCase{\lang}{
    {E}{
    %versão inglês
      \TITLE{\authorname}
        {\thesistitleen %\\[5mm]
        }
  \vspace*{5mm}
  \TEXT{}
       {\textsf{\Large{\thesistitle}} \\[5mm]
       \ifodd \norient{} 
       {\tipologiaDPE{\documenttype}à Universidade de Aveiro para cumprimento dos requisitos necessários à obtenção do grau de \thesisdegreept{} em \sciarea, realizada sob orientação científica de \supervisor{}, \supdegree{}, do \supfili{} da Universidade de Aveiro.}
       \else 
       {\tipologiaDPE{\documenttype}à Universidade de Aveiro para cumprimento dos requisitos necessários à obtenção do grau de \thesisdegreept{} em \sciarea, realizada sob orientação científica de \supervisor{}, \supdegree{}, do \supfili{} da Universidade de Aveiro, e de \cosupervisor{}, \cosupdegree{} do \cosupfili{} da \cosupuni{}. } 
       \fi
       } 
    }
    {P}{
    %versão português
      \TITLE{\authorname}
        {\thesistitle %\\[5mm]
        }
  \vspace*{5mm}
  \TEXT{}
       {\textsf{\Large{\thesistitleen}} \\[5mm]
       \ifodd \norient{} 
       {\tipologiaDPE{\documenttype}à Universidade de Aveiro para cumprimento dos requisitos necessários à obtenção do grau de \thesisdegreept{} em \sciarea, realizada sob orientação científica de \supervisor{}, \supdegree{}, do \supfili{} da Universidade de Aveiro.}
       \else 
       {\tipologiaDPE{\documenttype}à Universidade de Aveiro para cumprimento dos requisitos necessários à obtenção do grau de \thesisdegreept{} em \sciarea, realizada sob orientação científica de \supervisor{}, \supdegree{}, do \supfili{} da Universidade de Aveiro, e de \cosupervisor{}, \cosupdegree{} do \cosupfili{} da \cosupuni{}. } 
       \fi
       }
    }
    }
    
    \vspace*{110mm}
  \TEXT{}
       {\small{\apoio}}
  
    \EndTitlePage
    
    
  
%%RM2022


\titlepage\ \endtitlepage                        % Página de verso em branco


% -------------------------------------------------------------
% Definição da página do júri da dissertação (PT/EN)
% -------------------------------------------------------------
\TitlePage
  \vspace*{55mm}
  \TEXT{\textbf{O júri~/~The jury\newline}}
       {}
  \TEXT{Presidente~/~President}
       {\textbf{\tribpresident}\newline {\small
        \tribpresidentcat{António Manuel Godinho Completo}}}
  \vspace*{5mm}
  \TEXT{Vogais~/~Committee}
       {\textbf{\tribi}\newline {\small
        \tribicat{António Manuel Godinho Completo} \textit{\tribifili}}}       
  \vspace*{5mm}
  \TEXT{}
  {\ifodd\orientjuri{}{\textbf{\profdoc{} \supervisor}\newline {\small
        \supdegree{} da Universidade de Aveiro (Orientador)}}
       \else{\textbf{\profdoc{} \cosupervisor}\newline {\small
        \cosupdegree{} da \textit{\cosupuniit{}} (co-orientador)}}
  \fi}
        
\EndTitlePage

\titlepage\ \endtitlepage                        % Página de verso em branco


% -------------------------------------------------------------
% Definição da página dos agradecimentos
% -------------------------------------------------------------
\TitlePage
  \vspace*{55mm}
  \TEXT{\textbf{Agradecimentos~/\newline Acknowledgements}}
      {\agradecimentos}
\EndTitlePage
\titlepage\ \endtitlepage                        % Página de verso em branco

% -------------------------------------------------------------
% Definição da página do resumo (EN)
% -------------------------------------------------------------
\TitlePage
   \vspace*{55mm}
   \TEXT{\textbf{Keywords}}
        {\keyword{}}
   \vspace*{5mm}
   \TEXT{\textbf{Abstract}}
        {\abstr{}}
\EndTitlePage
\titlepage\ \endtitlepage                        % Página de verso em branco


% -------------------------------------------------------------
% Definição da página do resumo (PT)
% -------------------------------------------------------------
\TitlePage
   \vspace*{55mm}
   \TEXT{\textbf{Palavras-chave}}
        {\palavraschave{}}
   \vspace*{5mm}
   \TEXT{\textbf{Resumo}}  
        {\resumo{}}
\EndTitlePage
\titlepage\ \endtitlepage                        % Página de verso em branco


% -------------------------------------------------------------
% Listas e definições iniciais da dissertação
% -------------------------------------------------------------
\pagenumbering{roman}          % Numeração romana para as primeiras páginas
\tableofcontents               % Índice da dissertação
\listoftables                  % Lista de tabelas
\listoffigures                 % Lista de figuras
% Nomenclature
\makenomenclature
\nomenclature{EU}{\qquad - Europe union}
\nomenclature{COP}{\qquad - Coefficient of performance}
\nomenclature{$\dot{Q}_\mathrm{L}$}{\qquad - Cooling capacity [W]}
\nomenclature{${W_{\mathrm{net,in}}}$ }{\qquad - Work input [W]}
\nomenclature{TES}{\qquad - Thermal Energy Storage}
\nomenclature{$Q$ }{\qquad - Total heat transferred}
\nomenclature{$m$}{\qquad - Mass }

\printnomenclature         % Lista de nomenclatura
\cleardoublepage
%\titlepage\ \endtitlepage      % Página de verso em branco
\pagenumbering{arabic}         % Numeração árabe para as páginas restantes


% ----------------------------------------------------------------
% Definição de headers e footers
% ----------------------------------------------------------------
\pagestyle{fancy}
\renewcommand{\chaptermark}[1]{\markboth{\thechapter.#1}{}} % Capítulos em minúsculas
\fancyhf{}                                                  % Reset aos headers e footers
  \fancyhead[LE,RO]{\thepage}                               % Header Left-Even (LE), Right-Odd (RO)
  \fancyhead[LO]{\leftmark}                                 % Header Left-Odd (LO)
  \fancyhead[RE]{\leftmark}                                 % Header Right-Even (RE)
  \fancyfoot[LE]{\authornam}                               % Footer Left-Even (LE)
  \fancyfoot[LO]{\authornam}                               % Footer Left-Odd (LO)
%  \fancyfoot[RE]{\textit{Dissertação de \thesisdegree}}     % Footer Right-Even (RE)
%  \fancyfoot[RO]{\textit{Dissertação de \thesisdegree}}     % Footer Right-Odd (RO)
  \fancyfoot[RE]{\textit{\thesisdegree\ Degree}}     % Footer Right-Even (RE)
  \fancyfoot[RO]{\textit{\thesisdegree\ Degree}}     % Footer Right-Odd (RO)
  \renewcommand{\headrulewidth}{0.25pt}                     % Espessura da linha de header
  \renewcommand{\footrulewidth}{0.25pt}                     % Espessura da linha de footer
  \addtolength{\headheight}{0.5pt}                          % Espaçamento para a linha
