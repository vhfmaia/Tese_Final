% -------------------------------------------------------------
% Definições do trabalho
% -------------------------------------------------------------
% Nome do autor da tese
\newcommand{\authorname}{Victor Hugo \\ Florêncio Maia}  
% Nome abreviado do autor da tese
\newcommand{\authornam}{Victor Maia}         
% Título do trabalho em Português (MAX 130 caracteres)
\newcommand{\thesistitle}{Estudo e projeto de uma ferramenta para tratamento térmico de carbonitruração}
% Título do trabalho em Inglês (MAX 130 chars)
\newcommand{\thesistitleen}{Study and design of a tool-piece carrier for carbonitriding thermal treatment}               
% Ano da tese
\newcommand{\thesisyear}{2023}                               


% -------------------------------------------------------------
% Tipo de trabalho (substituir "D" por "P"=projeto ou "E"=Estágio)
% -------------------------------------------------------------
\newcommand{\documenttype}{E}          



% -------------------------------------------------------------
% Equipa de orientação
% -------------------------------------------------------------
\newcommand{\norient}{1}  %definir 1 se apenas orientador ou 2 se tiver co-orientador

% ORIENTADOR
%  > nome orientador
\newcommand{\supervisor}{António Manuel Godinho Completo}
%  > grau académico orientador
\newcommand{\supdegree}{Professor Associado com Agregação}
%  > filiação orientador
\newcommand{\supfili}{Departamento de Engenharia Mecânica}

% CO-ORIENTADOR
% > nome
\newcommand{\cosupervisor}{}           % Co-orientador
% > grau académico coorientador
\newcommand{\cosupdegree}{}
% > filiação do co-orientador
\newcommand{\cosupfili}{}  
% > instituição do co-orientador
\newcommand{\cosupuni}{}      
%  > instituição do co-orientador (original)
\newcommand{\cosupuniit}{}


% -------------------------------------------------------------
% Preâmbulo do documento - Membros do júri
% -------------------------------------------------------------

%Presidente do júri
% > nome presidente
\newcommand{\tribpresident}{}
% > categoria presidente
\newcommand{\tribpresidentcat}{}


% arguente do júri
% > nome do arguente
\newcommand{\tribi}{}
% > categoria do arguente
\newcommand{\tribicat}{}     
% filiação do arguente
\newcommand{\tribifili}{}

% elemento da equipa de orientação
\newcommand{\orientjuri}{1} % colocar 1 se orientador no júri ou 2 se co-orientador
\newcommand{\preorientjuri}{\profdoc{}} % substituir \profdoc por doctor se o membro for investigador


% -------------------------------------------------------------
% Preâmbulo do documento - páginas iniciais
% -------------------------------------------------------------

\newcommand{\agradecimentos}
{A caminhada até aqui não foi fácil, mas cada passo foi valioso. Não posso deixar os primeiros agradecimentos para ninguém além da minha família. Aos meus pais, Esdras e Marilene o vosso amor e apoio incondicional foram a minha rocha. Ao meu irmão, Vinicius, obrigado por me recordar o valor da persistência e pelo teu apoio. Por fim, à minha avó, Veneranda, os seus conselhos foram determinantes na minha construção como pessoa.
\par
Um agradecimento sincero também ao Professor Dr. António Manuel Godinho Completo. As suas análises e seu incentivo foram um suporte essencial para a conceção desta tese.
\par
Devo igualmente um agradecimento aos engenheiros Tiago Castelão e Alvaro Pires. A vossa ajuda, conhecimentos e conselhos foram determinantes na construção desta tese.
\par
À minha turma de colegas, vocês tornaram este percurso muito mais agradável. O apoio mútuo e a troca de ideias foram verdadeiramente valiosos. Quero também agradecer a todos que trabalham no Departamento de Engenharia Mecânica da Universidade de Aveiro, a vossa ajuda foi imprescindível.
\par
Por último, mas com certeza não menos importante, um obrigado especial aos meus amigos. Os jantares, os momentos de descontração e a vossa presença nos momentos mais difíceis foram um bálsamo necessário nesta jornada. A vossa amizade é um tesouro que espero nunca perder.
\par
Agradeço-vos a todos por fazerem parte desta etapa tão importante da minha vida. Este trabalho é o resultado de todos estes esforços e apoios. Muito obrigado!
}

\newcommand{\palavraschave}
{Carbonitruração, Têmpera, Ferramenta de Proteção, Fissuração à frio
}

\newcommand{\keyword}
{Carbonitriding, Quenching, Protecting tool-piece, Cold Cracking
}

\newcommand{\resumo}
{
A presente tese de mestrado foca-se na área de tratamentos termoquímicos e design e conceção de produto, com um estudo específico sobre Carbonitruração. Este trabalho aborda a conceção de uma ferramenta para a proteção do diâmetro interno de rodas de coroa num tratamento de carbonitruração, e os impactos desta proteção nos parâmetros destes componentes.
\par
Iniciando com uma revisão abrangente da literatura, esta investigação destaca as principais teorias e pesquisas existentes que deram forma ao nosso entendimento atual dos tratamentos termoquímicos. Esta tese nasce de uma necessidade da empresa Renault Cacia em aprimorar o processo produtivo com uma ferramenta porta-peças que não se encontra na literatura atualmente.
\par
Através de uma análise por uma simulação CFD, esta tese analisa o mecanismo que cria uma geometria protegida nas rodas de coroa, o que permite a conceção de uma ferramenta nos parâmetros estudados. Com acesso aos resultados, é concebido e estudado um protótipo da ferramenta proposta, feitos ensaios e os parâmetros obtidos são analisados, comparando os resultados com os parâmetros ja existentes.
\par
Os principais resultados da investigação indicam que o tamponamento do diâmetro interno das rodas de coroa traz enormes benefícios nos processos de torneamento duro e soldadura em caixas diferenciais. Estas descobertas têm implicações significativas para a indústria automobilística, sugerindo que caso seja implementada, este tipo de ferramenta pode ser fulcral no tratamento deste tipo de componentes no futuro.
}

\newcommand{\abstr}
{
This master's degree thesis focuses on the area of thermochemical treatments and product design and conception, with a specific study on Carbonitriding. This work addresses the design of a tool for the protection of the inner diameter of crown wheels in a carbonitriding treatment, and the impacts of this protection on the parameters of these parts.
\par
Starting with a comprehensive literature review, this research highlights the main existing theories and research that have shaped our current understanding of thermochemical treatments. This thesis arises from a need by the Renault Cacia company to improve the production process with a workpiece carrier tool that is not currently found in the literature.
\par   
Through an analysis by CFD simulation, this thesis analyses the mechanism that creates a protected geometry on the crown wheels, which allows the design of a tool within the parameters studied. With access to the results, a prototype of the proposed tool is designed and studied, tests are made and the obtained parameters are analysed, comparing the results with the already existing parameters.
\par 
The main results of the research indicate that protecting the inner diameter of the crown wheels brings enormous benefits in hard turning and the welding process in differential cases. These findings have significant implications for the automotive industry, suggesting that if implemented, this type of tool could be central to the treatment of this type of component in the future.
}

\newcommand{\apoio}
{
}