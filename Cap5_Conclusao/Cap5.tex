\chapter{Conclusão} \label{ch:conclusao}
\setlength{\headheight}{13.6pt}
%%%%%%%%%%%%%%%%%%%%%%%%%%%%%%%%%%%%%%%%%%%%%%%%%%%%%
Chegamos, finalmente, às conclusões deste documento.
\par
Conclui-se, portanto, que a carbonitruração desempenha um papel importante na indústria automobilística. Este projeto, realizado no contexto de um estágio curricular, revelou-se extremamente valioso para a aquisição de conhecimentos sobre o processo produtivo, o funcionamento de uma empresa multinacional e as relações interpessoais envolvidas.
\par
No que diz respeito ao projeto da ferramenta porta-peças para o tratamento térmico por carbonitruração, verificou-se que o tamponamento realizado abaixo e acima de uma coluna de rodas dentadas a ser tratada, embora não impeça o enriquecimento de carbono (C) e azoto (N) no forno de carbonitruração, limita significativamente a transferência de calor por convecção entre o fluido de têmpera e as rodas de coroa, evitando praticamente por completo o efeito de têmpera. Além disso, o uso desse tipo de ferramenta não influencia de forma significativa as durezas finais dos dentados, conforme verificado no Capítulo \ref{ch:resultados}. Também se constatou que a geometria externa da tampa de proteção superior exerce pouca influência nos parâmetros finais, o que permite maior liberdade na seleção da geometria da ferramenta, adequando-a às limitações geométricas do processo. No entanto, a inclusão de uma aba para proteção da cible da roda de coroa no topo revela-se uma adição importante, contribuindo para a uniformização das microestruturas e durezas das áreas que posteriormente serão torneadas e soldadas, o que facilita a seleção e a padronização das ferramentas a serem utilizadas.
\par
Como perspetivas de trabalhos futuros, recomendamos a realização de um ensaio para verificar a quantidade de hidrogénio difusível, a fim de determinar se ocorre penetração de hidrogénio no forno de carbonitruração. No entanto, é possível antecipar uma redução das não conformidades decorrentes de fissurações causadas pelo hidrogénio, uma vez que a presença de martensite na microestrutura final do diâmetro interno é reduzida, resultando em menor propensão à captura de hidrogénio. Após a realização do ensaio de hidrogénio difusível e caso os resultados sejam favoráveis, poderá ser considerada a redução do tempo de ciclo do revenido, uma vez que um dos objetivos desse processo é diminuir a quantidade de hidrogénio aprisionado durante a carbonitruração. Além disso, é recomendável otimizar o processo de torneamento duro levando em consideração os novos níveis de dureza do material a ser maquinado, com possibilidade de redução do tempo de ciclo por meio do aumento da profundidade de corte e da velocidade de corte, ou ainda considerando a substituição das ferramentas de corte por materiais mais económicos que atendam às novas exigências do material.
\par
Por fim, caso haja interesse, a implementação de uma ferramenta porta-peças nessas configurações exigirá o desenvolvimento de um sistema de transporte para as tampas, uma vez que, conforme abordado neste trabalho, essas tampas devem ser removidas antes do processo de revenido.