%%%%%%%%%%%%%%%%%%%%%%%%%%%%%%%%%%%%%%%%%%%%%%%%%%%%%%%%%%%%%%%%%%%%%%%%%%%%%
\chapter{Introdução} \label{ch:intro}
\setlength{\headheight}{13.6pt}
%%%%%%%%%%%%%%%%%%%%%%%%%%%%%%%%%%%%%%%%%%%%%%%%%%%%%%%%%%%%%%%%%%%%%%%%%%%%%

%%%%%%%%%%%%%%%%%%%%%%%%%%%%%%%%%%%%%%%%%%%%%%%%%%%%%%%%%%%%%%%%%%%%%%%%%%%%%
\section{Contextualização} \label{s:intro_contextualizacao}

Na soldadura de caixas diferenciais, um dos principais defeitos é a ocorrência de fissurações a frio, que são causadas quando há uma \textbf{microestrutura suscetível}, uma \textbf{fonte de hidrogénio difusível} e \textbf{tensões residuais provenientes do processo de soldadura}. Na Renault Cacia, as rodas de coroa estão especialmente sujeitas à este defeito por sofrerem um processo de carbonitruração, fulcral para a obtenção das propriedades pretendidas para suas aplicações, no entanto, neste processo ocorrem duas das tres grandes causas de fissuração a frio. Uma vez que o processo de carbonitruração utiliza como fontes de Carbono e Azoto, respetivamente, o \textbf{Propano (C3H8)} e o \textbf{Amoníaco (NH4)}, ambos ricos em hidrogénio; e o processo de têmpera realizado após o enriquecimento transforma a microestrutura em martensite, que é extremamente suscetível à ocorrência de fissuração a frio, a ocorrência deste defeito de soldadura é extremamente recorrente no processo de produção dos diferenciais.
Para além disso, os altos valores de dureza provenientes do enriquecimento de carbono da carbonitruração e o conseguinte aumento de dureza por meio deste e da têmpera, com grande formação de martensite, obrigam que a etapa de torneamento após o tratamento térmico seja realizado com ferramentas equipadas com pastilhas de CBN, de altíssima dureza e alto custo, quando comparados com ferramentas de corte convencionais.
%%%%%%%%%%%%%%%%%%%%%%%%%%%%%%%%%%%%%%%%%%%%%%%%%%%%%%%%%%%%%%%%%%%%%%%%%%%%%
\section{Objetivos da tese} \label{s:intro_objetivos}

Este trabalho tem então como objetivo o estudo, projeto e desenvolvimento de uma ferramenta porta-peças a ser utilizada no suporte das “Coroas” (roda dentada do diferencial) das caixas de velocidades \textbf{DB35}, \textbf{DB45} e \textbf{JT4}, durante o processo de tratamento termoquímico de Carbonitruração, com o objetivo de diminuir o enriquecimento e aumento de dureza obtidos no processo, de forma a facilitar o torneamento das "Coroas", possibilitando a diminuição dos custos nesta etapa, e diminuir a probabilidade de ocorrência de fissurações a frio, diminuindo a quantidade de não conformidades resultantes do processo de soldadura.
\par
\newpage
Para alcançar esses objetivos, neste trabalho é incluída a análise do processo de carbonitruração e das propriedades do material resultantes desse processo. Também é importante considerar as especificações do produto final, incluindo os requisitos de dureza e tolerâncias dimensionais, e desenvolver uma solução de suporte que atenda a essas especificações. O projeto também deve incluir a seleção de materiais para a ferramenta porta-peças e a avaliação dos custos associados à sua produção e uso. Por fim, é importante avaliar a eficácia da solução proposta por meio de simulações e ensaios, garantindo que esta atende aos objetivos estabelecidos.
%%%%%%%%%%%%%%%%%%%%%%%%%%%%%%%%%%%%%%%%%%%%%%%%%%%%%%%%%%%%%%%%%%%%%%%%%%%%%
\section{A Empresa} \label{s:intro_empresa}
O estagio em questão foi desenvolvido na Renault Cacia

%%%%%%%%%%%%%%%%%%%%%%%%%%%%%%%%%%%%%%%%%%%%%%%%%%%%%%%%%%%%%%%%%%%%%%%%%%%%%
\section{Organização do documento} \label{s:intro_organizacao}
O presente documento está dividido em cinco partes: Introdução, Revisão Bibliográfica, Materiais e Métodos, Análise dos Resultados e Conclusão.
\vspace{2mm}
\newline
O capítulo \ref{ch:intro}, \textbf{Introdução}, tem como objetivo a introdução e contextualização do problema, bem como a apresentação dos objetivos e uma síntese da metodologia utilizada para alcançar os objetivos descritos. 
\newline
O capítulo \ref{ch:soa}, \textbf{Revisão Bibliográfica}, tem como objetivo dar a conhecer o material técnico-cientifico existente sobre as áreas abordadas neste documento.
\newline
No capítulo \ref{ch:materiais}, \textbf{Materiais e Métodos}, é apresentado o caso de estudo em seu estado atual, as mudanças propostas, bem como todo o processo de conceção da proposta final de ferramenta.
\newline
No capítulo \ref{ch:resultados}, \textbf{Análise de Resultados}, são analisados os parâmetros de conformidade das peças de série em comparação com os mesmos parâmetros das peças com diâmetro protegido.
\newline
Por fim, o capítulo \ref{ch:conclusao}, \textbf{Conclusão}, resume as conclusões significativas do trabalho e fornece uma perspetiva dos trabalhos futuros, necessários para a implementação da ferramenta proposta.

%%%%%%%%%%%%%%%%%%%%%%%%%%%%%%%%%%%%%%%%%%%%%%%%%%%%%%%%%%%%%%%%%%%%%%%%%%%%%