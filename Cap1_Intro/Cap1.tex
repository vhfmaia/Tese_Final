%%%%%%%%%%%%%%%%%%%%%%%%%%%%%%%%%%%%%%%%%%%%%%%%%%%%%%%%%%%%%%%%%%%%%%%%%%%%%
\chapter{Introdução} \label{ch:intro}
\setlength{\headheight}{13.6pt}
%%%%%%%%%%%%%%%%%%%%%%%%%%%%%%%%%%%%%%%%%%%%%%%%%%%%%%%%%%%%%%%%%%%%%%%%%%%%%

%%%%%%%%%%%%%%%%%%%%%%%%%%%%%%%%%%%%%%%%%%%%%%%%%%%%%%%%%%%%%%%%%%%%%%%%%%%%%
\section{Contextualização} \label{s:intro_contextualizacao}

Na prática de soldadura aplicada às caixas diferenciais, um dos defeitos mais prevalentes é a ocorrência de fissurações por hidrogénio. Este fenómeno surge quando há a ocorrência  de três fatores: a presença de uma microestrutura vulnerável, uma fonte de hidrogénio difusível, e tensões residuais que são um subproduto do processo de soldadura.

Na Renault Cacia, as rodas de coroa são particularmente propensas a este defeito. Isso deve-se ao facto de serem submetidas a um processo de carbonitruração, essencial para alcançar as propriedades desejadas para as suas aplicações. No entanto, durante este processo, ocorrem duas das três principais causas de fissuração por hidrogénio.

O processo de carbonitruração emprega Propano (C3H8) e Amoníaco (NH4) como fontes de Carbono e Azoto, respetivamente, sendo ambos os compostos ricos em hidrogénio. O processo de têmpera subsequente à carbonitruração resulta na transformação da microestrutura em martensite, um material altamente suscetível à fissuração por hidrogénio. Estes fatores combinados resultam numa frequência considerável deste defeito de soldadura no processo de produção dos diferenciais.

Além disso, os valores elevados de dureza resultantes do enriquecimento de carbono no processo de carbonitruração e pela têmpera, com extensa formação de martensite, implicam que a etapa de torneamento após o tratamento térmico deva ser realizada com ferramentas equipadas com pastilhas de CBN. As pastilhas de CBN são de elevada dureza e custo, o que as distingue das ferramentas de corte convencionais.
%%%%%%%%%%%%%%%%%%%%%%%%%%%%%%%%%%%%%%%%%%%%%%%%%%%%%%%%%%%%%%%%%%%%%%%%%%%%%
\section{Objetivos da tese} \label{s:intro_objetivos}
Este estudo visa a conceção, desenvolvimento e implementação de uma ferramenta porta-peças para apoiar as rodas de coroas das caixas diferenciais das caixas de velocidades JT4, DB45 e DB35 durante o processo de tratamento termoquímico de Carbonitruração. O objetivo principal é reduzir o enriquecimento e aumento de dureza obtidos no processo, de forma a facilitar o torneamento das rodas de coroas, o que permitirá a redução de custos nesta etapa, bem como a diminuição da probabilidade de ocorrência de fissurações por hidrogénio, resultando numa menor quantidade de não conformidades no processo de soldadura.

Para alcançar estes objetivos, foi realizada uma análise detalhada do processo de carbonitruração e das propriedades materiais resultantes deste processo. As especificações do produto final, incluindo os requisitos de dureza e tolerâncias dimensionais, foram igualmente tidas em consideração, o que permitiu o desenvolvimento de uma solução de suporte que atendesse a essas especificações. A eficácia da solução proposta foi posteriormente avaliada através de simulações e testes, garantindo que estes objetivos fossem atingidos.

%%%%%%%%%%%%%%%%%%%%%%%%%%%%%%%%%%%%%%%%%%%%%%%%%%%%%%%%%%%%%%%%%%%%%%%%%%%%%
\section{A Empresa} \label{s:intro_empresa}
O estágio de investigação foi realizado na Renault Cacia, uma empresa de renome localizada em Aveiro, Portugal. Com uma história rica e uma reputação sólida, a empresa tem desempenhado um papel fundamental no sector automóvel desde a sua fundação.

A Renault Cacia, sendo uma das principais unidades de produção da Renault na Europa, é responsável pelo fabrico de várias séries de caixas de velocidades e bombas de óleo. É reconhecida pelo desenvolvimento de soluções inovadoras, destinadas a atender às exigências em constante evolução do mercado automóvel. A empresa destaca-se pelo seu compromisso com a excelência e pela busca contínua por avanços tecnológicos, mantendo uma visão orientada para a sustentabilidade e a eficiência energética nos seus processos de produção.

Adicionalmente, a Renault Cacia tem um papel social importante na região de Aveiro, uma vez que contribui para o desenvolvimento local e proporciona empregos de qualidade à comunidade. Com uma visão orientada para o futuro e um compromisso inabalável com a inovação, a Renault Cacia continua a ser uma referência no sector industrial e automóvel em Portugal.

%%%%%%%%%%%%%%%%%%%%%%%%%%%%%%%%%%%%%%%%%%%%%%%%%%%%%%%%%%%%%%%%%%%%%%%%%%%%%
\section{Organização do documento} \label{s:intro_organizacao}
O documento está estruturado em cinco secções principais: Introdução, Revisão Bibliográfica, Materiais e Métodos, Análise dos Resultados e Conclusão.

Na secção \ref{ch:intro}, \textbf{Introdução}, o problema é introduzido e contextualizado, os objetivos são apresentados e uma síntese da metodologia utilizada para alcançar os objetivos é descrita.

A secção \ref{ch:soa}, \textbf{Revisão Bibliográfica}, tem como objetivo apresentar o conhecimento técnico-científico existente sobre as áreas abordadas neste documento.

Na secção \ref{ch:materiais}, \textbf{Materiais e Métodos}, o caso de estudo é apresentado no seu estado actual, as mudanças propostas são discutidas e todo o processo de conceção da proposta final de ferramenta é descrito.

A secção \ref{ch:resultados}, \textbf{Análise de Resultados}, analisa os parâmetros de conformidade das peças de série em comparação com os mesmos parâmetros das peças com diâmetro protegido.

Finalmente, a secção \ref{ch:conclusao}, \textbf{Conclusão}, resume as principais conclusões do trabalho e fornece uma perspetiva dos trabalhos futuros, necessários para a implementação da ferramenta proposta.
%%%%%%%%%%%%%%%%%%%%%%%%%%%%%%%%%%%%%%%%%%%%%%%%%%%%%%%%%%%%%%%%%%%%%%%%%%%%%